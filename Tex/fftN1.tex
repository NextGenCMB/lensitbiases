\documentclass[11pt]{article}

\usepackage[utf8]{inputenc}
\usepackage{mathrsfs}
\usepackage{euscript}
\usepackage{epsfig}
\usepackage{graphics}
\usepackage{graphicx}
\usepackage{amsmath}
\usepackage{amssymb}
%\usepackage{timestamp}
\usepackage{bm}
\usepackage[usenames,dvipsnames,svgnames,table]{xcolor}
\usepackage{xspace}
\usepackage{wasysym}
\usepackage{times}
\usepackage{appendix}
\usepackage{lipsum}
\usepackage[nolist,nohyperlinks]{acronym}
\usepackage{float}
\usepackage{subcaption}
\usepackage{simplewick}
\usepackage{tabularx}
\usepackage{booktabs}
%

\begin{document}

\newcommand{\bl}{\boldsymbol{l}}
\newcommand{\br}{\boldsymbol{r}}
\newcommand{\hn}[0]{{\hat n}}

\newcommand{\bll}{\boldsymbol{L}}
\newcommand{\intL}{\int_{\substack{\bl_1 + \bl_2 \\ =\bll }}}
\newcommand{\intLp}{\int_{\substack{\bl'_1 + \bl'_2 \\ =\bll' }}}

\newcommand{\cred}[1]{\color{red} {#1} \color{black}}
Starting from the 'usual' n1,
\begin{equation}
\begin{split}
n^{(1)}(L) = &\intL \intLp \Xi_{ij}(\bl_1, \bl_2) \Xi_{pq}(\bl'_1, \bl'_2) \\ &\times\left[ C^{\phi \phi}_{|\bl_1+\bl'_1|}f_{ip}(\bl_1, \bl'_1) f_{jq}(\bl_2, \bl'_2) + C^{\phi \phi}_{|\bl_1+\bl'_2|}f_{iq}(\bl_1, \bl'_2) f_{jp}(\bl_2, \bl'_1) \right].
\end{split}
\end{equation}
generally, the strategy used here is as follows (we use $S,T$ for Stokes indices $\in (I, Q, U)$ and $X, Y$ (or $I, J$) for $T, E, B$:
\begin{itemize}
	\item Write the estimators $W^{XY}$ and $W^{IJ}$ in terms of $I,Q,U$
	\item The responses then just brings down a factor $i \bl_\alpha C^{ST}_{\bl} + \rm{sym}$.
\end{itemize}

So we get:
\begin{equation}\label{eq:rfftN1}\boxed{
 \int d\br \:	\xi^{\alpha \beta}(\br) \left[2W^{(ST, (\alpha, \beta))}_{\bll}(\br)W^{(ST)}_{-\bll}(\br) + 2 W^{(ST, \alpha,)}_{\bll}(\br)W^{(ST, (,\beta))}_{-\bll}(\br)\right]}
\end{equation}
where there is a sum (trace) over the Stokes map S, T, and ($X, Y \in (T, E ,B), S, T \in (T, Q, U)$). With $F$ the filtering matrix, $C^w$ the QE weights spectra, we have $(\bl_1 + \bl_2 = \bll)$
\begin{equation}
	W^{ST}_{\bl_1,\bl_2} = \frac 12 (FC^{w})^{SS'}_{\bl_1} F_{\bl_2}^{S'T} \cdot (\bl_1\cdot \bll) + \frac 12 F^{SS'}_{\bl_1} (C^wF)_{\bl_2}^{S'T} \cdot (\bl_2\cdot \bll)\quad (\textrm{gradient})
\end{equation}
or
\begin{equation}
	W^{ST}_{\bl_1,\bl_2} = \frac 12 (FC^{w})^{SS'}_{\bl_1} F_{\bl_2}^{S'T} \cdot (\bl_1\star \bll) + \frac 12 F^{SS'}_{\bl_1} (C^wF)_{\bl_2}^{S'T} \cdot (\bl_2\star \bll)\quad (\textrm{curl})
\end{equation}
and 
\begin{equation}
	W^{ST}(\br) = \int \frac{d\bl}{2\pi} e^{i \bl \cdot \br} \:W^{ST}_{\bll /2 +\bl, \bll/2 - \bl} 
\end{equation}
The functions with upperscripts $(\alpha,)$ just inserts a $i{\bl}{_1}^{\alpha} C^{f}_{\bl_1}$ on the left, or on the right if $(,\alpha)$.
We define $(-1)^P$ the sign of flipping the two components of $\bl_1$ and $\bl_2$; the lensing gradient has $(-1)^P = 1$, the lensing curl has $(-1)^P = -1$.
\subsection{Symmetries}
We then use symmetries to fasten the implementation, reducing the number of terms to compute
\begin{itemize}
	\item First, we get rid of $-\bll$ and replace it with $\bll$, using (valid for all QE keys)
\begin{equation}
\begin{split}
  W_{-\bll}^{ST} &= W_{+\bll}^{TS}\\
	W_{-\bll}^{ST, (\alpha,\beta)} &= W_{+\bll}^{TS, (\beta,\alpha)}, \\
	W_{-\bll}^{ST, (,\alpha)} &= -W_{+\bll}^{TS, (\alpha,)} \\
	\end{split}
	\end{equation}
\item We may also use that taking the conjugate is the same as flipping the sign of $\bl$, which has the effect of switching $\bl_1$ and $\bl_2$. The weight functions obeys $W^{ST}_{\bl_1\bl_2} = W_{\bl_2\bl_1}^{TS}$, so that
\begin{equation}
\begin{split}
	\left[W_{\bll}^{ST}\right]^*(\br) &= W_{\bll}^{TS}(\br) \\
	\left[W_{\bll}^{ST, (\alpha, \beta)}\right]^*(\br) &= W_{\bll}^{TS, (\beta, \alpha)}(\br) \\
		\left[W_{\bll}^{ST, (\alpha,)}\right]^*(\br) &= -W_{\bll}^{TS, (, \alpha)}(\br) \\
\end{split}
\end{equation}
\item Then, by setting $\bll = (L,L)/\sqrt{2}$ we can turn some $\beta$ derivatives onto $\alpha$ with the help of matrix transposes. Swapping the first and second coordinates one has
\begin{equation}
\begin{split}
	W_{\bll}^{ST}(\br)& = \textrm{sgn}\cdot W^{ST}_{\bll}({\br}^t) \\
	W_{\bll}^{ST, (1,)}(\br)& = \textrm{sgn}\cdot  W_{\bll}^{ST, (0,)}(\br^t) \\
		W_{\bll}^{ST, (1, 1)}(\br)& = \textrm{sgn}\cdot  W_{\bll}^{ST, (0,0)}(\br^t) \\
				W_{\bll}^{ST, (0,1)}(\br)& = \textrm{sgn}\cdot  W_{\bll}^{ST, (1,0)}(\br^t) \\
		\textrm{ with sgn} &= (-1)^P (-1)^{S == Q} (-1)^{T == Q},\quad  (\bll = (L,L)/\sqrt{2})\end{split}
\end{equation}
This assumes all $TB$ and $EB$ spectra are zero, and $\bll$ of the special form   $\bll = (L,L)\sqrt{2}$. This comes about because under a swap of first and second coordinate of $\bl$, we have $\sin 2\phi_{\bl}$ invariant, but $\cos 2\phi_{\bl}$ takes a minus sign, and the underlying weight functions takes a $(-1)^P$.
\end{itemize}
Using these results in the following expression for the $TS$ contribution in Eq.\ref{eq:rfftN1}, less elegant but with less calculations to perform:
\begin{equation}
\begin{split}
n^{1} \ni	&\int d\br \: 2\xi^{00}(\br) \cdot \left[2W_{\bll}^{TS, (0, 0)}(\br) W_{\bll}^{ST}(\br)- 2 W_{\bll}^{TS, (0,)}(\br)W_{\bll}^{ST, (0,)}(\br)\right] \\
	+&\int d\br \: 2\xi^{01}(\br)\cdot\left[ 2 W_{\bll}^{TS, (0,1)}(\br) W_{\bll}^{ST}(\br) - 2\textrm{sgn}\cdot W_{\bll}^{TS, (0,)}(\br)W_{\bll}^{ST, (0,)}(\br^t)\right]
\end{split}
\end{equation}
(we also used there $\xi^{01}(\br) = \xi^{10}(\br) = \xi^{01}(\br^t)$ and $\xi^{00}(\br) = \xi^{11}(\br^t)$)
We have then for the diagonal terms $T=S$, (since $W^{TT}$ real)
\begin{equation}
\begin{split}
	n^{1} \ni	&\int d\br \: 2\xi^{00}(\br) \cdot \left[2W_{\bll}^{TT, (0, 0)}(\br) W_{\bll}^{TT}(\br)- 2 \left[W_{\bll}^{TT, (0,)}(\br)\right]^2\right] \\
	+&\int d\br \: 2\xi^{01}(\br)\cdot\left[ 2 \Re\left[{W_{\bll}^{TS, (0,1)}}\right](\br) W_{\bll}^{TT}(\br) - 2\textrm{sgn}\cdot W_{\bll}^{TT, (0,)}(\br)W_{\bll}^{TT, (0,)}(\br^t)\right]
	\end{split}
\end{equation}
This leads to $5$ rFFTs, so $3\times 5$ for the full MV diags. The off-diagonal $ST + TS $ contributions:
\begin{equation}
\begin{split}
n^{1} \ni	&\int d\br \:2 \xi^{00}(\br)\left[ 4\Re \left[2W_{\bll}^{TS, (0, 0)}(\br) W_{\bll}^{ST}(\br)\right]- 4 W_{\bll}^{TS, (0,)}(\br)W_{\bll}^{ST, (0,)}(\br)\right] \\
	+&\int d\br \: 2\xi^{01}(\br)\cdot\left[ 4\Re\left[ W_{\bll}^{TS, (0,1)}(\br) W_{\bll}^{ST}(\br)\right] - 4\textrm{sgn}\cdot W_{\bll}^{TS, (0,)}(\br)W_{\bll}^{ST, (0,)}(\br^t)\right]
\end{split}
\end{equation}
This leads to $10 $ rFFTs, so $3\times 10$ for the full MV, so 45 total including the diagonal terms.

\subsection{Others}
The other matrices are: $R_{\bl}$ the transfer matrix from TEB to TQU:
\begin{equation} R^{SX}_{\bl} = 
	\begin{pmatrix}
		1 & 0 & 0 \\ 0 & \cos 2\phi_{\bl} & -\sin 2\phi_{\bl} \\ 0 & \sin 2\phi_{\bl} & \cos 2\phi_{\bl}
	\end{pmatrix}_{SX}
\end{equation}
and $(R^tR)^{XY}_{\bl_1\bl_2}$
\begin{equation}
		\begin{pmatrix}
		1 & 0 & 0 \\ 0 & \cos 2\phi_{\bl_1\bl_2}= c_1c_2 + s_1 s_2 & \sin 2\phi_{\bl_1\bl_2}= -c_1s_2 + s_1 c_2 \\ 0 & -\sin 2\phi_{\bl_1\bl_2}= c_1s_2 - s_1 c_2 & \cos 2\phi_{\bl_1\bl_2}= c_1c_2 + s_1 s_2
	\end{pmatrix}_{XY}
\end{equation}
\end{document}
